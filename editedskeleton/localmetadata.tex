\title{The lexeme in descriptive and theoretical morphology}  
\BackBody{Since the 1970s, the notion of a lexeme, an abstract lexical unit identifying what is common to a set of words belonging to the same inflectional paradigm, has become a cornerstone of theoretical thinking on morphology and a standard tool for description. The present volume collects papers that crucially use, discuss or question the lexeme in the context of contemporary morphology, with particular emphasis on its place in the description of word formation through the concept of a \emph{Lexeme Formation Rule}. 
It will be of interest to any descriptive linguist, theoretical linguist, or psycholinguist with an interest in morphology and its interface with syntax and lexical semantics. }
%\typesetter{Change typesetter in localmetadata.tex}
%\proofreader{Change proofreaders in localmetadata.tex}
\author{Olivier Bonami \and Gilles Boyé  \and Georgette Dal  \and Hélène Giraudo \lastand  Fiammetta Namer}
% \BookDOI{}%ask coordinator for DOI
\renewcommand{\lsISBNdigital}{000-0-000000-00-0}
\renewcommand{\lsISBNhardcover}{000-0-000000-00-0}
\renewcommand{\lsISBNsoftcover}{000-0-000000-00-0} 
\renewcommand{\lsSeries}{eotms} % use lowercase acronym, e.g. sidl, eotms, tgdi
\renewcommand{\lsSeriesNumber}{4} %will be assigned when the book enters the proofreading stage
% \renewcommand{\lsURL}{http://langsci-press.org/catalog/book/000} % contact the coordinator for the right number
\renewcommand{\lsID}{165} 
